% Chapter 1 - THEORY

\chapter{Theory} % Main chapter title

\label{chap:theory} % For referencing the chapter elsewhere, use \ref{Chapter1} 

\lhead{Chapter 1. \emph{Established theory}} % This is for the header on each page - perhaps a shortened title

%----------------------------------------------------------------------------------------
\section{Basis of LSFEM}
Let us look at a general boundary value problem where $f \in Y(\Omega)$, $g \in B(\partial \Omega)$, $B:X(\partial \Omega) \rightarrow B(\partial\Omega) $ and $L:X(\Omega)\rightarrow Y(\Omega)$. Find $u \in X(\Omega) $ such that 
\begin{align}
	Lu &= f \; \; \; \text{ in } \Omega \\
	Bu &= g \; \; \; \text{ on } \partial \Omega.
	\label{eq:BVP}
\end{align}
Whenever this BVP has a unique solution, a least-squares functional can be defined as 
\begin{align}
	J(u;f,g) = ||Lu-f||^2_Y + ||Bu-g||^2_B
	\label{eq:FunctionalGen}
\end{align}
and the corresponding minimization problem is then given as 
\begin{align}
	\min_{u \in X}J(u;f,g)
	\label{eq:minProbGen}
\end{align}
For any well-posed problem $\exists \alpha,\beta > 0$ such that 
\begin{align}
	\beta||u||_X \leq J(u;0,0) \leq \alpha||u||_X.
	\label{eq:normEq}
\end{align}
The fact that our functional is norm-equivalent is of crucial importance to a successfull LS-method.  
Minimizing this functional is equivalent to solving the Euler-lagrange equations formulated as 
\begin{align}
	\text{find } u \in X \text{  such that  } Q(u,v) = F(v) \; \; \forall v\in X
	\label{eq:varFormGen}
\end{align}
Where $Q(u,v)$ is the continous bilinear form given as $(Lu,Lv)_Y$ and $F(v)$ is the bounded linear functional given as $(Lv,f)_Y$.

\section{Example - Poisson problem}

The poisson problem is defined as 
\begin{align}
	-\Delta u = f \text{ in } \Omega \\
	u = g \text{ on } \partial \Omega
	\label{eq:Poisson}
\end{align}
The straight forward LSFEM approach is to define $w = -\nabla u$ and solve the system of equations 
\begin{align}
	w + \nabla u = 0 \text{ in } \Omega \\
	\nabla w = f \text{ in } \Omega \\
	u = 0 \text{ on } \partial \Omega.
	\label{eq:PoissonSystem}
\end{align}
which can be written in the same form as ~\ref{eq:BVP} with $ u = w \oplus u $, $f = (0,0,f)$, $g=0$, $B = (0,0,1)^T $ and $L$ given as 
\begin{align}
	L =
	\begin{bmatrix}
		1 & 0 & \partial / \partial x  \\
		0 & 1 & \partial / \partial y  \\
		\partial / \partial x & \partial/ \partial y  & 0
	\end{bmatrix}
	\label{eq:Amatrix}
\end{align}
We define the search space $X =  H^1(\Omega;\text{div}) \times H_0^1(\Omega)$ and the solution space $Y \times B  = [L^2(\Omega)]^3\times L^2(\Omega) $ and the functional can then be defined as in~\ref{eq:FunctionalGen}. The variational formulation of the problem can be stated. Find $ u \in X $ s.t.
\begin{align}
	Q(u,\phi) = F(\phi) \;\; \forall \;\; \phi \in X.
	\label{eq:VariationalFormulationPoisson}
\end{align}
We require that $f \in Y$.
Notice that the spaces $X$ and $Y$ chosen as described above fullfill the condition ~\ref{eq:normEq}. 

%
\section{Example - Diffusion convection problem}
%
The diffusion convection problem to be analyzed is given as 
\begin{align}
	-\Delta u + b \cdot \nabla u = f \text{ in } \Omega \\
	u = g \text{ on } \partial \Omega
	\label{eq:DiffTrans}
\end{align}
The straight forward LSFEM approach is to define $w = -\nabla u$ and solve the system of equations 
\begin{align}
	w + \nabla u = 0 \text{ in } \Omega \\
	\nabla \cdot w - b \cdot w = f \text{ in } \Omega \\
	u = 0 \text{ on } \partial \Omega.
	\label{eq:DiffTransSystem}
\end{align}
which can be written in the same form as ~\ref{eq:BVP} with $ z = w \oplus u $, $f = (0,0,f)$, $B = (0,0,1)^T $ and A given as 
\begin{align}
	A =
	\begin{bmatrix}
		1 & 0 & \partial / \partial x  \\
		0 & 1 & \partial / \partial y  \\
		\partial / \partial x - b_1 & \partial/ \partial y -b_2 & 0
	\end{bmatrix}
	\label{eq:AmatrixDiff}
\end{align}
By defining the residual and functional as in ~\ref{eq:Residual} and ~\ref{eq:Functional} the variational formulation of the problem can be stated. Find $ z \in W \times H_0^1$ s.t.
\begin{align}
	B(z,\phi) = (F,\phi) \;\; \forall \;\; \phi \in W \times H_0^1.
	\label{eq:VariationalFormulation}
\end{align}
With $W := \left\{ w \in \left[L_2(\Omega)\right]^{n_d}\right\}$.

\section{basis due to jiang}
\cite{Jiang}
The Least Squares Finite Element Method is a numerical method similar to mixed galerkin, however it assures a symmetric problem. Let us look at a system of first order differential equations on the form 
\begin{align}
	Au &= f \text{ in } \Omega \\
	Bu &= g \text{ on } \partial \Omega.
	\label{eq:PDE}
\end{align}
Where $A$ is a partial differential operator defined as 
\begin{align}
	A = \sum_{i = 1}^{n} A_i\frac{\partial}{\partial x_i} + A_0.
	\label{def:operatorA}
\end{align}
$n$ beeing the number of dimensions of the domain $\Omega$. Let us initially assume that $g=0$. Further we require $f \in L_2(\Omega)$ and choose $V = \left\{ v\in L_2(\Omega) | v = 0 \text{ on } \partial \Omega \right\}$. A residual is defined
\begin{align}
	R(v) = Av-f,
	\label{eq:Residual}
\end{align}
and a functional
\begin{align}
	 J(v) = \frac{1}{2}||R(v)||^2_0.
	\label{eq:Functional}
\end{align} 
The solution $u$ is restricted to the space $H^1_0(\Omega)$. By minimizing $J$ we obtain 
\begin{align}
	\lim_{t\rightarrow 0} \frac{d}{dt}I(u+tv) = \int_{\Omega}(Av)^T(Au-f)d\Omega = 0 \text{    ,   } \forall v \in V.
	\label{eq:minProb}
\end{align}
We can now write a variational formulation of the least-squares method: Find $u \in V$ such that 
\begin{align}
	B(u,v) = F(v) \; \; \; , \; \; \; \forall v \in V,
	\label{def:varForm}
\end{align}
where
\begin{align}
	B(u,v) = (Au,Av), \\
	F(v) = (f,Av).
	\label{def:bilin}
\end{align}
~\cite{Jiang}
Notice that the bilinear form $B$ is symmetric. The bilinear form that surged from a first-order problem by the LSFEM leads us to a variational formulation similar to the one obtained from a second order problem by regular FEM. Generally the bilinear form from LSFEM will correspond to a bilinear form of a problem of twice the order obtained using FEM. In order to avoid problems of large complexity a higher order PDE should therefore be transformed to a system of first order PDE's before applying the LSFEM-method. 

In order to apply a numerical algorithm the domain $\Omega$ needs to be discretized, we name this discretization $\Omega_h$. A set of basis functions $ \left\{ N \right\}_i $ is defined for $V_h = H^1_0(\Omega_h)$ such that the discrete variational formulation can be stated. Find $u_h \in V_h$ such that 
\begin{align}
	B(u_h,v_h) = F(v_h) \; \; \; , \; \; \; \forall v_h \in V_h,
	\label{def:varForm}
\end{align}
