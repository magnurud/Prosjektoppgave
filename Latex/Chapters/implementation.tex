% Chapter 3 - IMPLEMENTATION 

\chapter{Implementation} % Main chapter title

\label{chap:Implementation} % For referencing the chapter elsewhere, use \ref{Chapter1} 

\lhead{Chapter 3. \emph{Implementation details}} % This is for the header on each page - perhaps a shortened title

%----------------------------------------------------------------------------------------

 For the general problem ~\ref{eq:BVP} the functional $Q$ will take the form 
\begin{align}
	Q(u,v)=\int_{\Omega}(\mathcal{L}v)^T(\mathcal{L}u)d\Omega.
	\label{eq:functionalInt}
\end{align}
Implementing $Q$ requires two sets of basis functions $\{N_i\}$ that describes the search and solution space. In this project assignment the search and solution space will be described by the the same set of basis functions which will depend on the method applied. $u$ is discretized as 
\begin{align}
	u_h = \sum_{I=0}^{K}a_IN_I.
	\label{eq:uDisc}
\end{align}
Since equation ~\ref{eq:varFormGen} requires equality for all test functions in the search space we simply solve the equation for each basis function. We are therefore left with a system of $K$ equations. Equation ~\ref{eq:functionalInt} can then be written for each test function as  
\begin{align}
	Q(u_h,N_I) &= \int_{\Omega}(\mathcal{L}N_I)^T(\mathcal{L}u_h)d\Omega \\
	&= \int_{\Omega}(\mathcal{L}N_I)^T(\mathcal{L}\sum_{J=1}^Ka_JN_J)d\Omega \\
	&= \sum_{J=1}^K\int_{\Omega}(\mathcal{L}N_I)^T(\mathcal{L}a_JN_J)d\Omega \\
	&= \sum_{J=1}^K\int_{\Omega}(\mathcal{L}N_I)^T(\mathcal{L}a_JN_J)d\Omega \\
	&= \sum_{J=1}^K\int_{\Omega}(\mathcal{L}N_I)^T(\mathcal{L}N_J)d\Omega \;\cdot a_J.
	\label{eq:varFormDisc}
\end{align}
The total system of equation for all test functions can then be written as a matrix equation 
\begin{align}
	Au = F.
	\label{eq:matrixEq}
\end{align}
Where $A_{I,J}=\int_{\Omega}(\mathcal{L}N_I)^T(\mathcal{L}N_J)d\Omega$.
For the poisson equation $A_{I,J}$ will be a 3-by-3 matrix on the form 

$A_{I,J} = \int_{\Omega}
\begin{bmatrix}
	N_IN_J + N_{I,x}N_{J,x} & N_{I,x}N_{J,y} & N_IN_{J,x} \\ 	
	N_{I,y}N_{J,x} &N_IN_J + N_{I,y}N_{J,y} &  N_IN_{J,y} \\ 	
	N_{I,x}N_J & N_{I,y}N_J & N_{I,x}N_{J,x} + N_{I,y}N_{J,y} \\ 	
	\label{mat:basicPoisson}
\end{bmatrix}
d\Omega$

Notice that by doing the splitting of variables we obtain a system of equations three times as big as if we were to solve the equation directly. 

\section{LSFEM for poisson}
\colorbox{yellow}{finite element space $X_h^1$ ? quadrature\ldots  }
\section{LS spectral method for poisson}

The spectral implementation is done using Gauss Lobatto nodes and quadrature and the lagrange functions based on the GL nodes as basis functions. 
Notice that the discrete solution $u_h$ consist of the discretizations of both $u \text{ and } w = \nabla u$. $u_h$ can be structured blockwise such that or nodewise. By choosing a blockwise representation the final system of equations can be written as 

$
\begin{bmatrix}
	A_{1,1} & A_{1,2} &	A_{1,3} \\ 	
	A_{2,1} & A_{2,2} & A_{2,3} \\ 	
	A_{3,1} & A_{3,2} & A_{3,3} \\ 	
\end{bmatrix}
\begin{bmatrix}
 u^h \\ 	
 w^h_1\\ 	
 w^h_2\\ 	
\end{bmatrix}
=
\begin{bmatrix}
 0 \\ 	
 0\\ 	
 F^h\\ 	
\end{bmatrix}
$.

Where each block $A_{i,j}$ corresponds to  calculating element $i,j$ in the matrix ~\ref{mat:basicPoisson} for all the indices $I,J$. In order to implement this matrix it is convenient to write it in a compact form using the kronecker tensor product. The components needed for this formulation is the $n \times n$ diagonal matrix $ W $ with the GLL-weigths along the diagonal and the $n \times n$ matrix $(L)_{i,j}= l_j'(x_i)$ where $l_j$ is the jth lagrange polynomial and $x_i$ is the ith node in either x or y direction. Note that the formulation is based on a grid of GLL-nodes in both x and y direction. 

$
A = 
\begin{bmatrix}
	W \otimes (L^TWL+W) & WL \otimes L^TW 		 &	W \otimes WL  \\ 	
	L^TW \otimes WL     & (L^TWL+W) \otimes W  &	WL \otimes W  \\ 	
	W \otimes L^TW		  & L^TW \otimes W       &  L^TWL \otimes W +	W\otimes L^TWL  \\ 	
\end{bmatrix}
$

\colorbox{yellow}{do I need to show how this is derivated?}

Similarly without the reformulation of the PDE and with regular galerkin formulation the stiffness matrix will simply be 
$A_{3,3} = W \otimes L^TWL+ L^TWL \otimes W$
