% Chapter 2 - NEW THEORY

\chapter{Stability properties of least-squares} % Main chapter title

\label{chap:newTheory} % For referencing the chapter elsewhere, use \ref{Chapter1} 

\lhead{Chapter 2. \emph{Stability properties }} % This is for the header on each page - perhaps a shortened title

%----------------------------------------------------------------------------------------
\section{Instability with the Galerkin formulation}

With regular Galerkin approach for the diffusion transport equation you end up with the variational formulation 
%
\begin{align}
	a(u,v) = (f,v) \; \; \; \forall v \in V.
	\label{eq:varFormDiffFEM}
\end{align}
%
where $V$ is some closed subspace of $H^1$, and the bilinear functional is given as 
%
\begin{align}
	\begin{split}
	a(u,v) &= \mu\int_{\Omega}\nabla u \cdot \nabla v \: d\Omega 
	+ \int_{\Omega} v (b \cdot \nabla u) d \Omega
	\end{split}
	\label{eq:bilinearFunctional}
\end{align}
%
We can define a new norm from this functional in the following manner
%
\begin{align}
\begin{split}
	a(u,u) &= \mu\int_{\Omega}\nabla u \cdot \nabla u \: d\Omega 
	+ \int_{\Omega}  b \cdot (u\nabla u) d \Omega\\
	       &= \mu||\nabla u||^2_0 
	+ \frac{1}{2}\int_{\Omega} b \cdot \nabla u^2 d \Omega \\
	       &= \mu||\nabla u||^2_0 
	- \frac{1}{2}\int_{\Omega} u^2  (\nabla \cdot b)  d \Omega \\
\end{split}
	\label{eq:bilinearFunctional}
\end{align}
%
Where we have used Greens theorem and the assumption that $u=0$ on the boundary in the last equality. Let us assume that the divergence of our vector field can be bounded within some interval, say  $\gamma_0 \leq -\frac{1}{2}\nabla \cdot b \leq \gamma_1 $, we can now make a lower and upper bound for the norm surging from the bilinear form 
%
\begin{align}
	\mu||\nabla u||^2_0 + \gamma_0||u||^2_0 \leq a(u,u) \leq \mu||\nabla u||^2_0 + \gamma_1||u||^2_0
	\label{eq:bilinearOperatorBounds}
\end{align}
%
It is clear that for negative $\gamma_0$ and sufficiently small $\mu$ the bilinear form is no longer coercive and thus our convergence requirements are no longer valid. Standard Galerkin method is therefore not a suited way to solve this type of problem. 
\section{Adding least squares in order to obtain stability}

Now remember from equation \eref{eq:normEq} that the least squares formulation guarantees us a coercive bilinear form, given that the BVP has a solution and that the search- and solution space are chosen correctly. For this particular problem we can find $\alpha,\beta$ such that $\alpha||u||_1^2 \leq Q(\mathbf{u},\mathbf{u}) \leq \beta||u||_1^2 $.
Now, let us define $\mathring{a}(\cdot,\cdot)$ and $\mathring{f}(\cdot)$ as the as the linear combinations of the linear and bilinear form surging from standard Galerkin and least-squares method. 
\begin{align}
	\begin{split}
	\mathring{a}(u,v) &= a(u,v) + \delta Q(\mathbf{u},\mathbf{\mathbf{v}}) \\
	 \mathring{f}(v) &= (f,v) + \delta F(\mathbf{\mathbf{v}}) 
	\end{split}
	\label{eq:GLS}
\end{align}
Let us study the coerciveness of the bilinear form $\mathring{a}(\cdot,\cdot)$
\begin{align}
	\begin{split}
	\mathring{a}(u,u) &\geq \mu ||\nabla u||_0^2+\gamma_0||u||_0^2+\delta \alpha ||u||^2_1 \\
	&\geq \mu ||\nabla u||_0^2+\gamma_0||u||_0^2+\delta \alpha ||u||^2_0 \\
	&\geq \mu ||\nabla u||_0^2+\mu ||u||_0^2 \\
	&= \mu ||u||^2_1
	\end{split}
	\label{eq:coercivity}
\end{align}
\colorbox{yellow}{Don't like the way Q has to be defined with $\mathbf{u}$}
In the third inequality we make the assumption that $\gamma_0+\delta \alpha \geq \mu $ in other words $\delta$, (the amount of smoothing from LS) has to be chosen such that $\delta \geq (\mu-\gamma_0)/\alpha$.

Using \eref{eq:VarFormLinForms} with homogeneous boundary conditions we obtain the identities $Q(\mathbf{u},\mathbf{u}) = (\mathcal{L}\mathbf{u},\mathcal{L}\mathbf{u})$ and $F(\mathbf{u}) = (\mathbf{f},\mathcal{L}\mathbf{u})$ we can derive the following stability result for our discrete solution of the variational formulation, 
	
\begin{align}
	\mathring{a}(u_h,u_h)_h \leq C ||f||^2
	\label{eq:stabilityResult}
\end{align}
The proof for a similar method can be found in~\cite{Quarteroni} Ch.12. 
Let us start by assuming that the Galerkin formulation provides a bilinear form that can be stated as earlier, $a(u_h,u_h)= \mu||\nabla u_h||^2_0 + \gamma||u_h||^2_0$, with $\gamma > 0$. 
\colorbox{yellow}{Could be done with negative gamma as well }

Proof:
\begin{align}
	\begin{split}
	\mathring{a}(u_h,u_h) &= \mathring{f}(u_h) \\
	&= (f,u_h) + \delta(\mathbf{f},\mathcal{L}\mathbf{u_h}) \\
	&= (\frac{1}{\sqrt{\gamma}}f,\sqrt{\gamma}u_h) + \delta(\mathbf{f},\mathcal{L}\mathbf{u_h}) \\
	&\leq ||\frac{1}{\sqrt{\gamma}}f||\:||\sqrt{\gamma}u_h||+\delta||\mathbf{f}||\:||\mathcal{L}\mathbf{u_h}|| \\
	&\leq \frac{1}{\gamma}||f||^2 + \frac{1}{4}\gamma||u_h||^2 + \delta||\mathbf{f}||^2 + \frac{\delta}{4}||\mathcal{L}\mathbf{u_h}||^2 \\
	&\leq \frac{1}{\gamma}||f||^2 + \delta||\mathbf{f}||^2 + \frac{1}{4}\gamma||u_h||^2 + \frac{1}{4}\mu||\nabla u_h||^2 + \frac{\delta}{4}||\mathcal{L}\mathbf{u_h}||^2 \\
	&= \frac{1}{\gamma}||f||^2 + \delta||f||^2 + \frac{1}{4}\mathring{a}(u_h,u_h).
	\end{split}
	\label{eq:derivationStabilityResult}
\end{align}

Which Allows us to determine the constant $C= \frac{4}{3}(\frac{1}{\theta}+\delta)$




%Let us derive an upper bound for the norm created by the combined bilinear form $\mathring{a}(\cdot,\cdot)$. 
%\begin{align}
	%\mathring{a}(u,u) &= \mathring{f}(u) \\
	%&= (f,u)+(\mathcal{L}u,f)\\
	%&\leq ||f||^2_0 ||u||^2_0 + ||\mathcal{L}u||^2_0 ||f||^2_0\\
	%&= (||u||^2_0 + ||\mathcal{L}u||^2_0) ||f||^2_0
	%\label{eq:GLS}
%\end{align}


\section{Estimation of the coercivity constant $\alpha$}

In order to determine the amount of smoothing acquired it is necessary to know the coercivity constant $\alpha$. In this section I will determine this constant for the Poisson problem on $\Omega = (0,1)^2$.

%from for our domain $\Omega=(0,1)^2$ we end up with showing that $|v|_1\geq \sqrt{1/2}||v||_1 \;\; \forall v \in H^1_0$. 
We start by proving the Poincar\'e inequality on our domain $ \Omega $, a similar proof can be found in ~\cite{Quarteroni}. Let $\mathbf{g} = \frac{1}{\sqrt{2}}[x,y]$ such that $k = \nabla \cdot \mathbf{g}=\frac{2}{\sqrt{2}}$ and let $u \in H_0^1$ then we can outline the following

\begin{align}
	\begin{split}
	||u||_0^2 &= k^{-1}\int_{\Omega}k|u(\mathbf{x})|^2d\Omega \\
	&= -k^{-1}\int_{\Omega}\mathbf{g}\cdot \nabla(|u(\mathbf{x})|^2)d\Omega \\
	&= -2k^{-1}\int_{\Omega}\mathbf{g}\cdot [u(\mathbf{x})\nabla(u(\mathbf{x}))]d\Omega \\
	&\leq 2k^{-1}||\mathbf{g}||_{\infty}||u||_0 \: |u|_1\\
	&\leq 2k^{-1}\frac{1}{\sqrt{2}}||u||_0 \: |u|_1
	\end{split}
	\label{eq:PoincareProof}
\end{align}
Dividing both sides with the $L^2$-norm leaves us with
\begin{align}
	|u|_1 \geq ||u||_0
	\label{eq:PoincareStatement}
\end{align}
Further we can show the following result by using the definition of $||\cdot||_1$
\begin{align}
	\begin{split}
	||u||_1 &= \sqrt{||u||_0^2+|u|_1^2}\\
	&	\leq \sqrt{2|u|_1^2}\\
	&= \sqrt{2}|u|_1
	\end{split}
	\label{eq:NormInequalitySobolev}
\end{align}

This result can be used to show equivalency of the norm surging from the bilinear functional $Q(\cdot,\cdot)$ and the $|| \cdot ||_1$-norm . 
By using the inequality obtained in the previous section we can make the following argument,
\begin{align}
	\begin{split}
	|||u||| 
	&=Q(\mathbf{u},\mathbf{u})^{1/2} 
	= (\mathcal{L}\mathbf{u},\mathcal{L}\mathbf{u})_0^{1/2} \\
	&= \sqrt{||\nabla u+\mathbf{w}||^2_0+||\nabla \cdot \mathbf{w}||^2_0}\\
	&\geq ||\nabla \cdot \mathbf{w}||_0
	= |\mathbf{w}|_1 \\
	&\geq \frac{1}{\sqrt{2}}||\mathbf{w}||_1\\
	&\geq \frac{1}{\sqrt{2}}||\mathbf{w}||_0
	= \frac{1}{\sqrt{2}}|u|_1\\
	&\geq \frac{1}{2}||u||_1
	\end{split}
	\label{eq:resultAlpha}
\end{align}
Hence $\alpha=\frac{1}{2}$. We have then proven coercivity and found the coercivity constant of the energy norm obtained from the least-squares formulation for the Poisson problem.  
\colorbox{yellow}{should find this constant for the difftrans problem}
%
