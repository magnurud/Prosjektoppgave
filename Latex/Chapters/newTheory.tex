% Chapter 2 - NEW THEORY

\chapter{New Theory} % Main chapter title

\label{chap:newTheory} % For referencing the chapter elsewhere, use \ref{Chapter1} 

\lhead{Chapter 2. \emph{New theory}} % This is for the header on each page - perhaps a shortened title

%----------------------------------------------------------------------------------------
\section{Using the Least squares method to gain stability}

In regular galerkin FEM for the Diffusion-Transport equation you end up with the variational formulation 
\begin{align}
	a(u,v) = (f,v)_0 \; \; \; \forall v \in V
	\label{eq:varFormDiffFEM}
\end{align}
where the bilinear functional is given as 
\begin{align}
	a(u,v) &= \mu\int_{\Omega}\nabla u \cdot \nabla v d\Omega 
	+ \int_{\Omega} v (b \cdot \nabla u) d \Omega
	\label{eq:bilinearFunctional}
\end{align}
We can define a new norm from this functional in the following manner
\begin{align}
	a(u,u) &= \mu\int_{\Omega}\nabla u \cdot \nabla u d\Omega 
	+ \int_{\Omega} u (b \cdot \nabla u) d \Omega\\
	       &= \mu||\nabla u||^2_0 
	+ \frac{1}{2}\int_{\Omega} b \cdot \nabla u^2 d \Omega \\
	       &= \mu||\nabla u||^2_0 
	- \frac{1}{2}\int_{\Omega} u^2  (\nabla \cdot b)  d \Omega \\
	\label{eq:bilinearFunctional}
\end{align}
Let $\gamma_0 \leq - \frac{1}{2}\nabla \cdot b \leq \gamma_1 $, we can now make a lower and upper bound for the bilinear form 
\begin{align}
	\mu||\nabla u||^2_0 + \gamma_0||u||^2_0 \leq a(u,u) \leq \mu||\nabla u||^2_0 + \gamma_1||u||^2_0
	\label{eq:bilinearOperatorBounds}
\end{align}

\colorbox{yellow}{should I change signs in the equation?? } 

It is clear that for small $\mu$ and large $\gamma_1$ the bilinear form is no longer coercive and thus our convergence requirements are no longer valid.Remember that the functional in the LSFEM-formulation is chosen such that it is norm-equivalent with $|| \cdot ||_1 $, hence we can find $\alpha,\beta$ such that $\alpha||u||_1^2 \leq Q(u,u) \leq \beta||u||_1^2 $.
Now, let us explore what happens if we add the variational formulation obtained from the LSFEM to the standard FEM formulation
\begin{align}
	a(u,v) + \delta Q(u,v) &= (f,v) + \delta F(v) \\
	\mathring{a}(u,v) &= \mathring{f}(v)
	\label{eq:GLS}
\end{align}
Let us study the coerciveness of this functional,
\begin{align}
	\mathring{a}(u,u) &\geq \mu ||\nabla u||_0^2+\gamma_0||u||_0^2+\delta \alpha ||u||^2_1 \\
	&\geq \mu ||\nabla u||_0^2+\gamma_0||u||_0^2+\delta \alpha ||u||^2_0 \\
	&\geq \mu ||\nabla u||_0^2+\mu ||u||_0^2 \\
	&= \mu ||u||^2_1
	\label{eq:coercivity}
\end{align}
In the third inequality we make the assumption that $\gamma_0+\delta \alpha \geq \mu $ in other words $\delta$, (the amount of smoothing from LS) has to be chosen such that $\delta \geq (\mu-\gamma_0)/\alpha$.
We can also prove stability for a discrete solution of the variatonal formulation, 
\begin{align}
	\mathring{a}(u_h,u_h)_h \leq C ||f||_0
	\label{eq:stabilityResult}
\end{align}
The proof for a similar method can be found in ~\cite{Quarteroni} Ch.12. 

%Let us derive an upper bound for the norm created by the combined bilinear form $\mathring{a}(\cdot,\cdot)$. 
%\begin{align}
	%\mathring{a}(u,u) &= \mathring{f}(u) \\
	%&= (f,u)+(\mathcal{L}u,f)\\
	%&\leq ||f||^2_0 ||u||^2_0 + ||\mathcal{L}u||^2_0 ||f||^2_0\\
	%&= (||u||^2_0 + ||\mathcal{L}u||^2_0) ||f||^2_0
	%\label{eq:GLS}
%\end{align}


